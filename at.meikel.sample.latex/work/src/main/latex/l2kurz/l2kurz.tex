% This is  L2KURZ.TEX - LaTeX2e Kurzbeschreibung v2.x, Erlangen 1998, 1999
% This was L2KURZ.TEX - LaTeX2e Kurzbeschreibung, Mainz 1994,1995
% This was LKURZ.TEX - LaTeX Kurzbeschreibung, Uni Graz & TU Wien, 1987.
% last changes: 1999-04-18 WaS
%
\newcommand{\lkver}{2.1}             % laufende Versionsnummer ...
\newcommand{\lkdate}{18. April 1999} % ... und Datum
%
% Um die LaTeX-Kurzbeschreibung zu formatieren, benoetigen Sie LaTeX2e
% (in der Version vom Juni 1998 oder neuer) und folgende Dateien:
%
%      l2kurz.tex (diese Datei),
%      l2k1.tex, l2k2.tex, l2k3.tex, l2ksym.tex, 
%      l2k4.tex, l2k5.tex, l2k6.tex, l2ka.tex, a.ps,
%      german.sty (Version 2.5d oder neuer), latexsym.sty,
%      textcomp.sty, alltt.sty, graphics.sty
%
% LaTeX2e muss ueber die "traditionellen" deutschen Trennmuster
% dehypht.tex (von N. Schwarz, Bochum, 1993/1994) verfuegen..
%
% Bitte senden Sie Aenderungswuensche und Hinweise auf Fehler an
% Walter Schmidt <walter.schmidt@arcormail.de>.
% ----------------------------------------------------------------------

\documentclass[11pt,a4paper]{article} % ergaenze `twoside', wenn gewunscht!
\NeedsTeXFormat{LaTeX2e}[1998/06/01]  % wegen \mathring und Textsymbolen

\typeout{   Copyright 1998, 1999 W.Schmidt, J.Knappen, H.Partl, I.Hyna   }
\typeout{   Copyright 1994, 1995 J.Knappen, H.Partl, E.Schlegl, I.Hyna   }
\typeout{   Copyright 1987 H.Partl, E.Schlegl, I.Hyna                    }

\usepackage{german,latexsym,alltt}
\usepackage[dvips,draft]{graphics} 
% Eine ps-Datei muss gelesen werden, aber die dvi-Datei soll mit jedem 
% dvi-Treiber, auch ohne PostScript-Unterstuetzung, zu verarbeiten sein!

% Sind die TC-Fonts vorhanden?
\newif\iftcfonts\tcfontstrue
\def\checkfont#1{%
  \batchmode
  \font\test=#1\relax
  \errorstopmode
  \ifx\test\nullfont
    \tcfontsfalse
  \fi}
\checkfont{tcrm1095}

% Wenn ja, dann lade textcomp.sty:
\iftcfonts\usepackage{textcomp}\fi

% Texthoehe 46 Zeilen + \topskip
% (Rand ueber der Kopfzeile) : (Rand unter dem Text) = 1:2
%
\setlength{\textheight}{46\baselineskip}
\addtolength{\textheight}{\topskip}
\newlength{\uppermargin}
\setlength{\uppermargin}{\paperheight}
\addtolength{\uppermargin}{-\textheight}
\addtolength{\uppermargin}{-\headsep}
\addtolength{\uppermargin}{-\headheight}
\setlength{\uppermargin}{.333\uppermargin} % Raender oben/unten 1:2
\addtolength{\uppermargin}{-1in}
\setlength{\topmargin}{\uppermargin}
%
% Textbreite 5.2in
% Text hor. zentriert
%
\setlength{\textwidth}{5.2in}
\setlength{\oddsidemargin}{.5339in}
\setlength{\evensidemargin}{\oddsidemargin}
%

% Seitenzahlen oben, aber keine Kopfzeile
\pagestyle{myheadings}
\markboth{}{}

%
\renewcommand{\textfraction}{.1}      % Make float placement easier
\renewcommand{\floatpagefraction}{.7}


\newcommand{\bs}{\symbol{92}} % Ein Backslash in Schreibmaschinenschrift;
                              % nur mit OT1-Fonts zu verwenden!
\makeatletter
%
% \path aus dtk.sty ( heisst dort \Path )
\begingroup
\gdef\path@SepI{/""}
\gdef\path@SepII{\symbol{92}""}
\gdef\path@SepIII{:""}
\catcode`\/=13
\catcode`\:=13
\catcode`\^=0
^catcode`\\=13
^gdef^path{^begingroup
  ^catcode`^/=13 
  ^catcode`^\=13 
  ^catcode`^:=13 
  ^catcode`^~=12 
  ^catcode`^$=12 %$
  ^catcode`^_=12 
  ^catcode`^#=12 
  ^let/=^path@SepI
  ^let\=^path@SepII
  ^let:=^path@SepIII
  ^@path}
^gdef^@path#1{^texttt{#1}^endgroup}
^endgroup
%
% LaTeXe-Symbol fuer cmss/sbc mit groesserem Absstand L-a und halbfettem Epsilon
\DeclareRobustCommand{\sbLaTeXe}{{\fontseries{sbc}\selectfont\boldmath%
        L\kern-.25em% -.36
        {\sbox\z@ T%
         \vbox to\ht\z@{\hbox{\check@mathfonts
                              \fontsize\sf@size\z@
                              \math@fontsfalse\selectfont
                              A}%
                        \vss}%
        }%
        \kern-.15em%
        \TeX\kern.15em2$_{\textstyle\varepsilon}$}}
%        
\makeatother

\newcommand\exa{\nopagebreak \begin{flushleft}\smallskip \nopagebreak
                \begin{minipage}[t]{6cm}\sloppy}
\newcommand\exb{\end{minipage}\kern 1cm\begin{minipage}[t]{8cm}\sloppy }
\newcommand\exc{\end{minipage}\kern -3cm \smallskip\end{flushleft}}
 
\newcommand\oben[1]{\begin{center}\begin{minipage}{#1}\hrule\medskip}
\newcommand\unten  {\hrule \end{minipage}\end{center}}

\newenvironment{ttdescription}{%
  \renewcommand{\descriptionlabel}[1]{%
    \hspace{\labelsep}\texttt{##1}}%
  \begin{description}%
}{%
  \end{description}%
}

\newcommand{\manual}{\emph{\LaTeX-Handbuch}~\cite{manual}}
\newcommand{\local}{\emph{Local Guide}~\cite{local}}

\newenvironment{symbols}{%
   \begin{tabbing}
   \hspace{1cm}\=\hspace{3.5cm}\=  \hspace{1cm}\=\hspace{3.5cm}\=
   \hspace{1cm}\=\hspace{3.5cm}\=  \kill
   }{%
   \end{tabbing}}

\newcommand{\nfrac}[2]{\leavevmode\kern.1em%
  \raise.5ex\hbox{\scriptsize #1}%
  \kern-.1em/\kern-.15em%
  \lower.25ex\hbox{\scriptsize #2}}

\nonfrenchspacing      % german.sty sets frenchspacing automatically.
%  However, some examples are pointless with frenchspacing in action. 
%  Besides, the larger space after a sentence make the text more readable.


\begin{document}

\begin{titlepage} % <-------- neue Titelseite
\renewcommand{\thefootnote}{\fnsymbol{footnote}}
{\Huge%
\fontfamily{cmss}\fontseries{sbc}\selectfont
\raggedright
\sbLaTeXe-Kurzbeschreibung
\rule{\textwidth}{0.75pt}
\par
}
\begin{flushleft}
  \normalsize
  \fontfamily{cmss}\fontseries{sbc}\selectfont
  Version \lkver\\
  \lkdate\\[2ex]
  Walter Schmidt\footnote{%
    Erlangen, \texttt{<walter.schmidt@arcormail.de>}}\\
  J"org Knappen\footnote{%
    Electronic Technologies, Springer-Verlag, Heidelberg}\\
  Hubert Partl\footnote{%
    Zentraler Informatikdienst der Universit"at f"ur Bodenkultur, Wien}\\
  Irene Hyna\footnote{%
    Bundesministerium f"ur Wissenschaft und Verkehr, Wien}
\end{flushleft}

\vfill

{\parindent=0cm
\LaTeX{} ist ein Satzsystem, das f"ur viele Arten von
Schrift"-st"u"cken verwendet werden kann, von einfachen Briefen bis zu
kompletten B"uchern.  Besonders geeignet ist es f"ur 
wissenschaftliche oder technische Dokumente. \LaTeX{} ist f"ur 
praktisch alle verbreiteten Betriebssysteme verf"ugbar.
 
Die vorliegende Kurzbeschreibung bezieht sich auf die Version
\LaTeXe\ in der Fassung vom 1.\ Juni~1998 und sollte f"ur den 
Einstieg in \LaTeX{} ausreichen.  
Eine voll"-st"andige Beschreibung ent"-h"alt das \manual{}
in Verbindung mit der Online-Dokumentation.
}
\setcounter{footnote}{0}
\end{titlepage}

%\newpage

{\parindent=0cm\thispagestyle{empty}
\copyright{} Copyright 1998, 1999 W.~Schmidt, J.~Knappen, H.~Partl, I.~Hyna
\bigskip

Die Verteilung dieses Dokuments in elektronischer oder gedruckter
Form ist gestattet, solange sein Inhalt einschlie"slich Autoren- und 
Copyright-Angabe unver"andert bleibt und die Verteilung kostenlos
erfolgt, abgesehen von einer Ge\-b"uhr f"ur den Datentr"ager, den
Kopiervorgang usw.
\bigskip

Die in dieser Publikation erw"ahnten Software- und Hardware-Bezeichnungen sind
in den meisten F"allen auch eingetragene Warenzeichen und unterliegen als
solche den gesetzlichen Bestimmungen.
\bigskip

\vfill

Dieses Dokument wurde mit \LaTeX{} gesetzt.
Sein Quelltext ist zu finden unter
\path{<ftp://ftp.dante.de/tex-archive/info/lshort/german/>}.
\bigskip

Die Autoren danken Michael Hofmann, Rainer Sch"opf, Stefan 
Steffens, Luzia Dietsche, Bernd Raichle und Heiko Oberdiek 
f"ur Tips, Anmerkungen 
und  Korrekturen.
}

\newpage

\setcounter{page}{1}
\tableofcontents

\newpage
 
\input{l2k1}
\clearpage

\input{l2k2}
\clearpage
 
\input{l2k3} 
\input{l2ksym}
\clearpage
 
\input{l2k4}
\clearpage

% master: l2kurz.tex
% L2K5.TEX - 5.Teil der LaTeX2e-Kurzbeschreibung v2.x, Erlangen 1998, 1999
% 1999-04-18 WaS

\section{Schriften}
Normalerweise w"ahlt \LaTeX\ die Gr"o"se und den Stil der Schrift
aufgrund der Befehle aus, die die logische Struktur des Textes angeben:
"Uber"-schriften, Fu"snoten, Hervorhebungen usw.
Im folgenden werden Befehle und Makropakete beschrieben, mit denen
die Schrift auch explizit beeinflu"st werden kann.
Ausf"uhrlichere Erl"auterungen zum Umgang mit Schriften in \LaTeX{}
findet man im \textit{\LaTeX-Begleiter} \cite{wonne} 
und in der Online-Dokumentation \cite{fntguide}.



\subsection{Schriftgr"o"sen}
 
Die in der Tabelle~\ref{sizes} an"-ge"-f"uhrten Befehlen 
wechseln die Schriftgr"o"se.
Sie spezifizieren die Gr"o"se relativ
zu der von \verb:\documentclass: festgelegten Grundschrift.
Ihr Wirkung reicht bis zum Ende der aktuellen Gruppe oder Umgebung.


\begin{table}[hb]
\caption{Schriftgr"o"sen} \label{sizes}
\oben{10cm}
\begin{tabbing}
\texttt{xfootnotesizexx}\= und dann der Text \kill
\verb|\tiny|         \> \tiny        winzig kleine Schrift \\
\verb|\scriptsize|   \> \scriptsize  sehr kleine Schrift (wie Indizes)\\
\verb|\footnotesize| \> \footnotesize     kleine Schrift (wie Fu"snoten)\\
\verb|\small|        \> \small            kleine Schrift \\
\verb|\normalsize|   \> \normalsize  normale Schrift \\
\verb|\large|        \> \large       gro"se Schrift \\
\verb|\Large|        \> \Large       gr"o"sere Schrift \\
\verb|\LARGE|        \> \LARGE       sehr gro"se Schrift \\[3pt]
\verb|\huge|         \> \huge        riesig gro"s \\[3pt]
\verb|\Huge|         \> \Huge        gigantisch
\end{tabbing}
\unten
\end{table}
 
Die Gr"o"sen-Befehle ver"andern auch die Zeilen"-ab"-st"ande auf
die jeweils passenden Werte -- aber nur, wenn die
Leerzeile, die den Absatz be\-en\-det, innerhalb des
G"ultigkeitsbereichs des Gr"o"sen-Befehls liegt:
\exa
{\Large zu enger\\
Abstand}\par
\exb
\begin{verbatim}
{\Large zu enger \\
Abstand}\par
\end{verbatim}
\exc
\exa
{\Large richtiger\\
Abstand\par}
\exb
\begin{verbatim}
{\Large richtiger\\
Abstand\par}
\end{verbatim}
\exc
F"ur korrekte
Zeilen"-ab"-st"ande darf die
schlie"-"sende geschwungene Klammer also nicht zu fr"uh kommen,
sondern erst nach einer Leerzeile oder einem explizit mit dem
Befehl~\verb|\par| ein"-ge"-f"ugten Absatz"-ende.


\subsection{Schriftstil}
Der Schriftstil wird in \LaTeX{} durch 3~Merkmale definiert:
\begin{description}
\item[Familie] Standardm"a"sig stehen 3~Familien zur Wahl:
  "`roman"' (Antiqua), "`sans serif"' (Serifenlose) und "`typewriter"'
  (Schreibmaschinenschrift).
\item[Serie] Die Serie gibt St"arke und Laufweite der
  Schrift an: "`medium"' (normale Schrift), "`boldface extended"'
  (fett und breiter).
\item[Form] Die Form der Buchstaben: "`upright"'
  (aufrecht), "`slanted"' (geneigt), "`italic"' (kursiv),
  "`caps and small caps"' (Kapit"alchen).
\end{description}
Tabelle~\ref{fonts} zeigt die Befehle, mit denen diese Attribute 
explizit beeinflu"st werden k"onnen.  
Die Befehle der Form \verb|\text...| setzen nur ihr Argument im 
gew"unschten  Stil.  Zu jedem dieser Befehle ist ein Gegenst"uck angegeben, 
das von seinem Auf\/treten an bis zum Ende der laufenden Gruppe oder Umgebung 
wirkt.

Zu beachten ist, da"s W"orter in Schreibmaschinenschrift nicht automatisch
getrennt werden.\par

\begin{table}[hbp]
\caption{Schriftstile} \label{fonts}
\oben{10cm}
\begin{tabbing}\small
\verb|\textnormal|\{\textit{text}\}\qquad\=\verb|\normalfont|\qquad\=\kill
\verb|\textrm|\{\textit{text}\}         \>\verb|\rmfamily|       \>\textrm{Antiqua}\\
\verb|\textsf|\{\textit{text}\}         \>\verb|\sffamily|       \>\textsf{Serifenlose}\\
\verb|\texttt|\{\textit{text}\}         \>\verb|\ttfamily|       \>\texttt{Maschinenschrift}\\[1ex]
\verb|\textmd|\{\textit{text}\}         \>\verb|\mdseries|       \>\textmd{normal}\\
\verb|\textbf|\{\textit{text}\}         \>\verb|\bfeseries|      \>\textbf{fett, breiter laufend}\\[1ex]
\verb|\textup|\{\textit{text}\}         \>\verb|\upshape|        \>\textup{aufrecht}\\
\verb|\textsl|\{\textit{text}\}         \>\verb|\slshape|        \>\textsl{geneigt}\\
\verb|\textit|\{\textit{text}\}         \>\verb|\itshape|        \>\textit{kursiv}\\
\verb|\textsc|\{\textit{text}\}         \>\verb|\scshape|        \>\textsc{Kapit"alchen}\\[1ex]
\verb|\textnormal|\{\textit{text}\}     \>\verb|\normalfont|     \>\textnormal{Die Grundschrift des Dokuments}
\end{tabbing}
\unten
\end{table}

Die Befehle f"ur Familie, Serie und Form k"onnen untereinander und mit den
Gr"o"sen-Befehlen kombiniert werden;  allerdings mu"s nicht jede
m"ogliche Kombination tats"achlich als reale Schrift (Font)
zur Verf"ugung stehen.
\exa
{\small Die kleinen
\textbf{fetten} R"omer
beherrschten }{\large das
ganze gro"se \textit{Italien}.}
\\[6ex]
{\Large\sffamily\slshape plakativ}
\exb
\begin{verbatim}
{\small Die kleinen
\textbf{fetten} R"omer
beherrschten }{\large das
ganze gro"se \textit{Italien}.}
{\Large\sffamily\slshape plakativ}
\end{verbatim}
\exc

Je \emph{weniger} verschiedene Schriftarten man verwendet, desto
lesbarer und sch"oner wird das Schrift"-st"uck!


\subsection{Andere Schriftfamilien}
Mit den im vorigen Abschnitt eingef"uhrten Befehlen kann man nicht beeinflussen,
welche Schriftfamilien tats"achlich als Antiqua, Serifenlose und
Maschinenschrift benutzt werden.  \LaTeX{} verwendet als Voreinstellung
die sog.\ Computer-Modern-Schriftfamilien (CM), siehe Tabelle~\ref{families};
der Stil der mathematischen Zeichens"atze pa"st dabei zu CM~Roman.

Will man andere Schriften benutzen, dann ist der einfachste Weg 
das Laden eines Pakets, das eine oder mehrere dieser Schriftfamilien 
komplett ersetzt.
Tabelle~\ref{families} f"uhrt einige derartige Pakete auf, 
die allerdings nicht mit jeder \LaTeX-Installation verf"ugbar sein m"ussen.
Bei den Schriftfamilien "`Times"', "`Palatino"', "`Helvetica"' und "`Courier"'
handelt es sich um Type-1-Fonts.
Ihre Verwendung setzt normalerweise voraus,
da"s die von \LaTeX{} erzeugte dvi-Datei zun"achst in das 
Post\-Script-Format umgewandelt wird,
bevor sie angezeigt oder ausgedruckt werden kann.  
In der Beschreibung \cite{local} Ihres \LaTeX-Systems sollte angegeben sein, 
ob und, wenn ja, wie das unterst"utzt wird.  
N"aheres zu Post\-Script-Schriften
finden Sie in \cite{wonne}, \cite{wonne-eng} und \cite{grfguide}.
\begin{table}[htb]
\caption[Pakete f"ur alternative Schriftfamilien]
{Pakete f"ur alternative Schriftfamilien (Eine leere
Tabellenspalte bedeutet, da"s das Paket die betreffende Schriftfamilie nicht 
ver"andert; * kennzeichnet die jeweils als Grundschrift eingestellte Familie.)}
\label{families}
{\footnotesize
\begin{center}
\medskip
\renewcommand{\arraystretch}{1.5}
\begin{tabular}{|l|p{2.cm}p{2.2cm}p{2.4cm}p{2.2cm}|}
\hline
Paket            & Antiqua    & Serifenlose   & Schreibmaschine  & math.\ Formeln\\\hline\hline
(keines)         & CM Roman * & CM Sans Serif & CM Typewriter    & $\approx$ CM Roman\\\hline
\texttt{ccfonts} & Concrete *
                 &
                 &
                 & $\approx$ Concrete\\\hline
\texttt{cmbright}&
                 & CM Bright *
                 & {\raggedright CM\ Typewriter\\ Light}
                 & $\approx$ CM Bright\\\hline
\texttt{pandora} & {\raggedright Pandora\\ Roman *} 
                 & {\raggedright Pandora \\ Sans Serif} 
                 &
                 & \\\hline
\texttt{mathptmx}& Times *
                 &
                 &
                 & $\approx$ Times\\\hline
\texttt{mathpple}& Palatino *
                 &
                 &
                 & $\approx$ Palatino\\\hline
\texttt{helvet}  & 
                 & Helvetica
                 &
                 & \\\hline
\texttt{courier} &
                 &
                 & Courier 
                 & \\\hline
\end{tabular}
\end{center}
}
\end{table}


\subsection{Die europ"aischen Schriften}
\LaTeX{} verwendet standardm"a"sig  Schriften mit einem Umfang von
128~Zeichen.  Umlaute oder akzentuierte Buchstaben sind darin nicht
enthalten; sie werden jeweils aus dem Grundsymbol und dem Akzent
zusammengesetzt.  Seit 1997 existieren zus"atzlich sogenannte
"`europ"aische"' Schriften.  Sie enthalten 256~Zeichen, welche fast
alle europ"aischen Sprachen abdecken, d.\,h., jedes ben"otigte
Zeichen ist vorgefertigt in ihnen enthalten.

Dies hat nicht nur eine
h"ohere typographische Qualit"at zur Folge; aufgrund der inneren Arbeitsweise
von \TeX{} entfallen damit auch die Einschr"ankungen im Zusammenhang mit
der Silbentrennung, die im Abschnitt~\ref{silb} erw"ahnt wurden:
W"orter mit Umlauten werden nun besser getrennt, und im Argument des
Befehls \verb|\hyphenation| d"urfen auch Umlaute und das scharfe~s stehen.
Weiterhin sind die Unterschneidungen im Vergleich zu den amerikanischen
\TeX-Originalschriften stark verbessert und nun auch auf h"aufige
Buchstabenpaarungen in nicht-englischen Sprachen optimiert.

% <------- Formulierung ????
Die europ"aischen Schriften bestehen aus zwei Teilen: Die T1-Schriften
enthalten Buchstaben, ASCII-Zeichen sowie verschiedene Anf"uhrungszeichen
und Striche, 
w"ahrend die TS1-Schriften zu\-s"atz\-liche Textsymbole bereitstellen.
% <------- Formulierung ????

\LaTeX{} wird veranla"st, T1-Schriften zu verwenden,
indem man das Paket \texttt{fontenc} mit der Option \texttt{T1} l"adt:
\begin{quote}
  \verb|\usepackage[T1]{fontenc}|
\end{quote}
Das Paket \texttt{textcomp} erm"oglicht den Zugriff auf die Textsymbole:
\begin{quote}
  \verb|\usepackage{textcomp}|
\end{quote}
Welche zus"atzlichen Zeichen mit den T1-Schriften
bereitgestellt werden, ist in \cite{usrguide} zusammengefa"st;
Anhang~\ref{textsymbols} der vorliegenden Kurzbeschreibung
enth"alt eine Liste aller TS1-Textsymbole.  Einige der Textsymbole sind
auch ohne das Paket \texttt{textcomp} verf"ugbar, siehe Abschnitt~\ref{symbole},
dann aber nicht immer in einem zur laufenden Schrift passenden Stil.


Alle Schriftfamilien der Tabelle~\ref{families}, 
mit Ausnahme von \texttt{pandora}, 
stehen sowohl mit dem standardm"a"sigen Zeichenvorrat
als auch in Form der europ"aischen Schriften zur Verf"ugung.  In den
PostScript-Schriften fehlen allerdings einzelne Zeichen, besonders aus
den Textsymbolen.

\endinput


\clearpage

\input{l2k6}
\clearpage

% master: l2kurz.tex
% L2KA.TEX - Anhang der LaTeX-Kurzbeschreibung v2.x, Erlangen 1998, 1999
% 1999-04-18 WaS

\appendix
\enlargethispage{1\baselineskip}

\section{Mit dem Paket \texttt{textcomp} verf"ugbare Symbole}
\label{textsymbols}

\iftcfonts % nur wenn die tc-Fonts vorhanden sind ...

\begingroup % We want to change the style of footnotes in this file only
\renewcommand{\thefootnote}{\fnsymbol{footnote}} 

{\small
\begin{tabbing}
\quad\quad\=\texttt{Mtextquotestraightdblbase}\hspace{1cm}\=\quad\quad\=\kill
\textquotestraightbase \> \verb+\textquotestraightbase+\footnotemark[1]  \> \textquotestraightdblbase \> \verb+\textquotestraightdblbase+\footnotemark[1] \\
\texttwelveudash \> \verb+\texttwelveudash+\footnotemark[1]  \> \textthreequartersemdash \> \verb+\textthreequartersemdash+\footnotemark[1] \\
\textleftarrow \> \verb+\textleftarrow+ \> \textrightarrow \> \verb+\textrightarrow+\\
\textblank \> \verb+\textblank+ \> \textdollar \> \verb+\$+\footnotemark[1] \\
\textquotesingle \> \verb+\textquotesingle+\footnotemark[1]  \> \textasteriskcentered \> \verb+\textasteriskcentered+\footnotemark[1] \\
\textdblhyphen \> \verb+\textdblhyphen+ \> \textfractionsolidus \> \verb+\textfractionsolidus+\footnotemark[1] \\
\textlangle \> \verb+\textlangle+ \> \textminus \> \verb+\textminus+\footnotemark[1] \\
\textrangle \> \verb+\textrangle+ \> \textmho \> \verb+\textmho+\\
\textbigcircle \> \verb+\textbigcircle+ \> \textohm \> \verb+\textohm+\\
\textlbrackdbl \> \verb+\textlbrackdbl+ \> \textrbrackdbl \> \verb+\textrbrackdbl+\\
\textuparrow \> \verb+\textuparrow+ \> \textdownarrow \> \verb+\textdownarrow+\\
\textasciigrave \> \verb+\textasciigrave+\footnotemark[1]  \> \textborn \> \verb+\textborn+\\
\textdivorced \> \verb+\textdivorced+ \> \textdied \> \verb+\textdied+\\
\textleaf \> \verb+\textleaf+ \> \textmarried \> \verb+\textmarried+\\
\textmusicalnote \> \verb+\textmusicalnote+ \> \texttildelow \> \verb+\texttildelow+\footnotemark[1] \\
\textdblhyphenchar \> \verb+\textdblhyphenchar+ \> \textasciibreve \> \verb+\textasciibreve+\footnotemark[1] \\
\textasciicaron \> \verb+\textasciicaron+\footnotemark[1]  \> \textacutedbl \> \verb+\textacutedbl+\footnotemark[1] \\
\textgravedbl \> \verb+\textgravedbl+\footnotemark[1]  \> \textdagger \> \verb+\dag+\footnotemark[1] \\
\textdaggerdbl \> \verb+\ddag+\footnotemark[1] \> \textbardbl \> \verb+\textbardbl+\footnotemark[1] \\
\textperthousand \> \verb+\textperthousand+\footnotemark[1] \> \textbullet \> \verb+\textbullet+\footnotemark[1] \\
\textcelsius \> \verb+\textcelsius+\footnotemark[1]  \> \textdollaroldstyle \> \verb+\textdollaroldstyle+\\
\textcentoldstyle \> \verb+\textcentoldstyle+ \> \textflorin \> \verb+\textflorin+\footnotemark[1] \\
\textcolonmonetary \> \verb+\textcolonmonetary+ \> \textwon \> \verb+\textwon+\\
\textnaira \> \verb+\textnaira+ \> \textguarani \> \verb+\textguarani+\\
\textpeso \> \verb+\textpeso+ \> \textlira \> \verb+\textlira+\\
\textrecipe \> \verb+\textrecipe+ \> \textinterrobang \> \verb+\textinterrobang+\\
\textinterrobangdown \> \verb+\textinterrobangdown+ \> \textdong \> \verb+\textdong+\\
\texttrademark \> \verb+\texttrademark+\footnotemark[1] \> \textpertenthousand \> \verb+\textpertenthousand+\\
\textpilcrow \> \verb+\textpilcrow+ \> \textbaht \> \verb+\textbaht+\\
\textnumero \> \verb+\textnumero+ \> \textdiscount \> \verb+\textdiscount+\\
\textestimated \> \verb+\textestimated+ \> \textopenbullet \> \verb+\textopenbullet+\\
\textservicemark \> \verb+\textservicemark+ \> \textlquill \> \verb+\textlquill+\\
\textrquill \> \verb+\textrquill+ \> \textcent \> \verb+\textcent+\footnotemark[1] \\
\textsterling \> \verb+\pounds+\footnotemark[1]  \> \textcurrency \> \verb+\textcurrency+\footnotemark[1] \\
\textyen \> \verb+\textyen+\footnotemark[1] \> \textbrokenbar \> \verb+\textbrokenbar+\footnotemark[1] \\
\textsection \> \verb+\S+\footnotemark[1]  \> \textasciidieresis \> \verb+\textasciidieresis+\footnotemark[1] \\
\textcopyright \> \verb+\copyright+\footnotemark[1] \> \textordfeminine \> \verb+\textordfeminine+\footnotemark[1] \\
\textcopyleft \> \verb+\textcopyleft+ \> \textlnot \> \verb+\textlnot+\footnotemark[1] \\
\textcircledP \> \verb+\textcircledP+ \> \textregistered \> \verb+\textregistered+\footnotemark[1] \\
\textasciimacron \> \verb+\textasciimacron+\footnotemark[1]  \> \textdegree \> \verb+\textdegree+\footnotemark[1] \\
\textpm \> \verb+\textpm+\footnotemark[1] \> \texttwosuperior \> \verb+\texttwosuperior+\\
\textthreesuperior \> \verb+\textthreesuperior+ \> \textasciiacute \> \verb+\textasciiacute+\footnotemark[1] \\
\textmu \> \verb+\textmu+\footnotemark[1] \> \textparagraph \> \verb+\P+\footnotemark[1] \\
\textperiodcentered \> \verb+\textperiodcentered+\footnotemark[1] \> \textreferencemark \> \verb+\textreferencemark+\\
\textonesuperior \> \verb+\textonesuperior+ \> \textordmasculine \> \verb+\textordmasculine+\footnotemark[1] \\
\textsurd \> \verb+\textsurd+ \> \textonequarter \> \verb+\textonequarter+\\
\textonehalf \> \verb+\textonehalf+ \> \textthreequarters \> \verb+\textthreequarters+\\
\textsf{\texteuro} \> \verb+\textsf{\texteuro}+ \> \texttimes \> \verb+\texttimes+\footnotemark[1] \\
\textdiv \> \verb+\textdiv+\footnotemark[1] \\
\end{tabbing}
}

{\footnotesize\noindent 
In PostScript-Schriften, die nicht speziell f"ur die Verwendung mit
\TeX{} entworfen wurden,  gibt es es normalerweise nur die mit 
\footnotemark[1]  markierten Zeichen.
\par}

\endgroup

\else
\par\vfill
\begin{quote}
\large\sffamily\slshape
Dieser Abschnitt konnte nicht gesetzt werden, weil die
CM-Schriften mit TS1-Codierung (tc-Fonts) nicht vorhanden sind!
\end{quote}

\vfill
\clearpage
\fi

\endinput


\begin{thebibliography}{99} \hbadness10000 % wg. URLs ;-)
 
\bibitem{manual}
L.~Lamport: \textit{Das \LaTeX-Handbuch.}
Addison-Wesley Deutschland (1995)%, ISBN~3-89319-826-1
. Deutsche "Uber"-setzung von~\cite{manual-eng}.

\bibitem{manual-eng}
L.~Lamport: \textit{\LaTeX, A Document Preparation System.}
Ad\-di\-son-Wesley, 2.~Aufl. (1994)%, ISBN~0-201-52983-1
.
 
\bibitem{wonne}
M.~Goossens, F.~Mittelbach und A.~Samarin:
\textit{Der \LaTeX-Begleiter.}
Ad\-di\-son Wesley Longman, 2.~korr.\ Nachdruck (1996)%, ISBN~3-89319-646-3
.  Deutsche "Uber"-setzung von~\cite{wonne-eng}.

\bibitem{wonne-eng}
M.~Goossens, F.~Mittelbach und A.~Samarin:
\textit{The \LaTeX\ Companion.}
Ad\-di\-son-Wesley (1994)%, ISBN~0-201-54199-8
.  

\bibitem{ch8}
M.~Goossens, F.~Mittelbach und A.~Samarin:
\textit{Higher Mathematics.} 
\path{<ftp://ftp.dante.de/tex-archive/info/companion-rev/ch8.pdf>} (1998).
Aktualisierte Fassung von Kapitel\ 8 aus \cite{wonne-eng}.

\bibitem{grfcomp}
M.~Goossens, S.~Rahtz und F.~Mittelbach:
\textit{The \LaTeX\ Graphics Companion.}
Addison Wesley Longman (1997)% ISBN~0-201-85469-4
.

\bibitem{local}
Zu jedem installierten \LaTeX-System sollte ein
\emph{\LaTeX\ Local Guide} vorhanden sein, in dem alle f"ur
dieses System spezifischen Angaben -- z.\,B.~die f"ur den
Aufruf der Programme notwendigen Befehle und die zur Ver"-f"ugung
stehenden Dokumentklassen, Pakete und Schriften -- angef"uhrt sind.
 
\bibitem{usrguide}
\LaTeX3 Project Team (Hrsg.): 
\textit{\LaTeXe\ for authors.} 
Bestandteil der Online-Dokumentation von \LaTeX,
Datei \texttt{usrguide.tex}.  
Aktuelle "Anderungen und Erg"anzungen sowie die Unterschiede zum fr"uheren 
\LaTeX~2.09 sind hier dokumentiert.

\bibitem{fntguide}
\LaTeX3 Project Team (Hrsg.): 
\textit{\LaTeXe\ font selection.}
Bestandteil der Online-Dokumentation von \LaTeX,
Datei \texttt{fntguide.tex}.

\bibitem{clsguide}
\LaTeX3 Project Team (Hrsg.): 
\textit{\LaTeXe\ for class and package writers.} 
Bestandteil der Online-Dokumentation von \LaTeX,
Datei \texttt{clsguide.tex}.

\bibitem{grfguide}
D.~P.~Carlisle: \textit{Packages in the `graphics' bundle.} 
Online-Dokumentation des \texttt{graphics}-Pakets,
Datei \texttt{grfguide.ps}.

\bibitem{epslatex}
K.~Reckdahl: \textit{Using Imported Graphics in \LaTeXe.} \\
\path{<ftp://ftp.dante.de/tex-archive/info/epslatex.ps>}

%\bibitem{postscript}
%S.~Rahtz: \textit{Notes on setup of PostScript fonts for \LaTeXe.} \\
%\path{<ftp://ftp.dante.de/tex-archive/macros/latex/required/psnfss/psnfss2e.tex>}

\bibitem{germdoc}
B.~Raichle:
\textit{Kurzbeschreibung -- \texttt{german.sty}.}
\path{<ftp://ftp.dante.de/tex-archive/language/german/gerdoc.tex>}


\bibitem{texbook}
D.~E.~Knuth: \textit{Computers \& Typesetting, Vol.\ A: The \TeX{}~Book.}
Addison-Wesley (1991)%, ISBN~0-201-13447-0
.

\bibitem{schwarz}
N.~Schwarz: \textit{Einf"uhrung in \TeX -- incl.\ Version~3.0.}
Oldenbourg, 3.~Aufl.\ (1991)%, ISBN 3-486-24349-7
.
 
\bibitem{germtug}
H.~Partl: \textit{German \TeX.} TUG\-boat Vol.~9, No.~1 (1988).
 
\bibitem{lay}
H.~Partl und A.~Kielhorn: \textit{Layout-"Anderungen mit \LaTeX.}
EDV-""Zentrum der Technischen Universit"at Wien,
\path{<ftp://ftp.dante.de/tex-archive/macros/latex/contrib/supported/refman/>}
(1996).

\bibitem{lay2}
D.~F.~Langmyhr: \textit{How to make your own document style in \LaTeXe.}
In: \textit{Proceedings of the Eighth European \TeX{} Conference}
(1994).

\bibitem{typografie} A.~Reichert: \textit{Typografie -- Gestaltung einer 
Beispielklasse.} \path{<ftp://ftp.dante.de/info/german/typografie/>} (1999).


\end{thebibliography}
 
\end{document}
