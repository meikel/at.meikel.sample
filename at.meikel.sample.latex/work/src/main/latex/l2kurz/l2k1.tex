% master: l2kurz.tex
% L2K1.TEX - 1.Teil der LaTeX2e-Kurzbeschreibung v2.x, Erlangen 1998, 1999
% L2K1.TEX - 1.Teil der LaTeX2e-Kurzbeschreibung Mainz 1994, 1995
% LK1.TEX  - 1.Teil der LaTeX-Kurzbeschreibung Graz-Wien 1987
% last changes: 1999-04-18 WaS


\section{Allgemeines}
 
% Note: the large number of short subsubsections in this section
% helps to avoid pages full of text (without any white space),
% which would look bad and might cause a `page memory overflow' on
% some printers like the XEROX 2700.
 
\subsection{The Name of the Game}
 
\subsubsection{\TeX}

\TeX\ (sprich "`Tech"', kann auch "`TeX"' geschrieben werden) ist
ein Computer\-progamm von Donald E.~Knuth~\cite{texbook,schwarz}.
Es dient zum Setzen und Drucken von Texten und mathematischen
Formeln.
 
\subsubsection{\LaTeX}
 
\LaTeX\ (sprich "`Lah-tech"' oder "`Lej-tech"', kann auch
"`LaTeX"' geschrieben werden) ist ein auf \TeX\ auf\/bauendes Computerprogramm
und wurde von Leslie Lamport~\cite{manual,wonne} geschrieben.
Es vereinfacht den Umgang mit \TeX, indem es 
entsprechend der logischen Struktur des Dokuments auf vorgefertigte
Layout-Elemente zur"uckgreift.

\subsubsection{\LaTeXe}

\LaTeXe\ (sprich "`\LaTeX\ zwei e"') ist die aktuelle Variante von
\LaTeX\ seit dem 1.~Juni 1994.  (Die vorherige hie"s \LaTeX~2.09.)
Wenn hier von \LaTeX\ gesprochen wird, so ist normalerweise dieses
\LaTeXe{} gemeint.

Neue Versionen  % <---------- bessere Formulierung?
von \LaTeXe{} (z.\,B. mit Fehlerberichtigungen oder Er\-g"an\-zun\-gen)
erscheinen zweimal j"ahrlich im Juni und im 
Dezember; die vorliegende Beschreibung setzt mindestens diejenige
vom Juni 1998 voraus.  

\subsection{Grundkonzept}
 
\subsubsection{Autor, Designer und Setzer}
 
F"ur eine Publikation "ubergab der Autor dem Verleger
traditionell  ein maschinengeschriebenes Manuskript.  Der
Buch-Designer des Verlages entschied dann "uber das Layout des
Schrift"-st"ucks (L"ange einer Zeile, Schriftart, Ab\-st"ande vor
und nach Kapiteln usw.\@) und schrieb dem Setzer die
daf"ur notwendigen Anweisungen dazu.
%--- Somit ist das gesetzte Schrift"-st"uck
%--- das Ergebnis des Buch-Designers und nicht des Autors.
\LaTeX{} ist in diesem Sinne der Buch-Designer, 
das Programm \TeX{} ist sein Setzer.
 
Ein menschlicher Buch-Designer erkennt die Absichten des Autors
(z.\,B.\ Kapitel-"Uberschriften, Zitate, Beispiele, Formeln
\dots) meistens aufgrund seines Fachwissens aus dem Inhalt des
Manuskripts.  \LaTeX{} dagegen ist "`nur"' ein Programm und
ben"otigt daher zus"atzliche Informationen vom Autor, die die
logische Struktur des Textes beschreiben.
Diese Informationen werden in Form von sogenannten "`Befehlen"'
innerhalb des Textes angegeben.
Der Autor braucht sich also
(weitgehend) nur um die logische Struktur seines Werkes zu k"ummern,
nicht um die Details von Gestaltung und Satz.
 
Im Gegensatz dazu steht der visuell orientierte Entwurf eines
Schrift"-st"u"ckes mit Textverarbeitungs- oder DTP-Programmen wie z.\,B.\ 
\textsc{Word}.
In diesem Fall legt der Autor das Layout des Textes gleich bei der
interaktiven Eingabe fest. Dabei sieht er am Bildschirm das, was
auch auf der gedruckten Seite stehen wird. Solche Systeme, die das
visuelle Entwerfen unter"-st"utzen, werden auch WYSIWYG-Systeme
("`what you see is what you get"') genannt.
 
Bei \LaTeX{} sieht der Autor beim Schreiben des Eingabe"-files in
der Regel noch nicht sofort, wie der Text nach dem Formatieren 
aussehen wird. Er kann aber %durch Aufruf des entsprechenden Programms 
jederzeit einen "`Probe-Ausdruck"' seines Schrift"-st"ucks auf dem
Bildschirm machen und danach sein Eingabe"-file entsprechend 
korrigieren und die Arbeit fortsetzen.
 
 
\subsubsection{Layout-Design}
 
Typographisches Design ist ein Handwerk, das erlernt werden mu"s.
Unge"-"ubte Autoren machen oft gravierende Formatierungsfehler.
F"alsch"-licherweise glauben viele Laien, da"s Textdesign
vor allem eine Frage der "Asthetik ist -- wenn das
Schrift"-st"uck vom k"unstlerischen Standpunkt aus "`sch"on"'
aussieht, dann ist es schon gut "`designed"'.
Da Schrift"-st"u"cke jedoch gelesen und nicht in einem Museum
auf"-ge"-h"angt werden, sind die leichtere Lesbarkeit und bessere
Ver"-st"and"-lichkeit wichtiger als das sch"one Aussehen.
 
Beispiele:
Die Schrift"-gr"o"se und Numerierung von "Uberschriften soll so
ge"-w"ahlt werden, da"s die Struktur der Kapitel und Unterkapitel
klar erkennbar ist.
Die Zeilen"-l"ange soll so ge"-w"ahlt werden, da"s anstrengende
Augenbewegungen des Lesers vermieden werden, nicht so, da"s der
Text das Papier m"oglichst sch"on aus"-f"ullt.
 
Mit interaktiven visuellen Entwurfssystemen ist es leicht,  
% "asthetisch sch"one, aber schlecht strukturierte
%-----  falsch: nur die Darstellung ist schlecht, 
%-----  die Strukturierung ist vielleicht ok.
Schrift"-st"u"cke zu erzeugen, die zwar "`gut"' aussehen,
aber ihren Inhalt und dessen Aufbau nur mangelhaft wiedergeben.
\LaTeX{} verhindert solche
Formatierungsfehler, indem es den Autor dazu zwingt, die logische
Struktur des Textes anzugeben, und dann automatisch ein da"-f"ur
geeignetes Layout verwendet.

Daraus ergibt sich, da"s \LaTeX{} insbesondere f"ur  Dokumente geeignet 
ist, wo vorgegebene Gestaltungsprinzipien auf sich wiederholende
logische Textstrukturen angewandt werden sollen. 
F"ur das -- notwendigerweise -- visuell orientierte Gestalten
etwa eines Plakates ist \LaTeX{} hingegen 
aufgrund seiner Arbeitsweise weniger geeignet.

\subsubsection{Vor- und Nachteile}

Gegen"uber anderen Textverarbeitungs- oder DTP-Programmen 
zeichnet sich \LaTeX{}
vor allem durch die folgenden Vorteile aus:
\begin{itemize}
\item Der Anwender mu"s nur wenige, leicht verst"andliche Befehle
  angeben, die die logische Struktur des Schrift"-st"ucks
  betreffen, und braucht sich um die gestalterischen Details
  (fast) nicht zu k"ummern.
\item Das Setzen von mathematischen Formeln ist besonders gut
  unter"-st"utzt.
\item Auch anspruchsvolle Strukturen wie Fu"snoten, Literaturverzeichnisse,
  Tabellen u.\,v.\,a.\  k"on"-nen mit wenig Aufwand erzeugt werden.
% ---- schwammige Formulierung ;-)
\item Routineaufgaben wie das Aktualisieren von Querverweisen
 oder die Erstellung des Inhaltsverzeichnisses 
 werden automatisch erledigt.
\item Es stehen zahlreiche vordefinierte Layouts zur Ver"-f"ugung.
\item \LaTeX-Dokumente sind zwischen verschiedenen Installationen und
 Rechnerplattformen austauschbar.
\item Im Gegensatz zu den meisten WYSIWIG-Programmen ist \LaTeX{} auch
  in Verbindung mit langen oder komplexen Dokumenten stabil,
  und sein Ressourcenverbrauch (Speicher, Rechenzeit) ist geringer.
\end{itemize}
Ein Nachteil soll freilich auch nicht verschwiegen werden:
\begin{itemize}
\item Innerhalb der von \LaTeX\ unter"-st"utzten Dokument-Layouts
  k"on"-nen zwar einzelne Parameter leicht variiert werden,
  grundlegende Abweichungen von den Vorgaben sind
  aber nur mit gr"o"-"serem Aufwand m"og"-lich (Design einer
  neuen Dokumentklasse, siehe~\cite{clsguide,lay,lay2,typografie}).
\end{itemize}

\subsubsection{Der Arbeitsablauf}
Der typische Ablauf beim Arbeiten mit \LaTeX{} ist:
\begin{enumerate}
  \item Ein Eingabefile schreiben, das den Text und die \LaTeX-Befehle 
  enth"alt.
  \item Dieses File mit \LaTeX{} bearbeiten; dabei wird eine Datei
  erzeugt, die den gesetzten Text in einem ger"ateunabh"angigen Format
  (dvi) enth"alt.
  \item Einen "`Probeausdruck"' davon auf dem Bildschirm anzeigen (Preview).
  \item Wenn n"otig, die Eingabe korrigieren und zur"uck zu Schritt~2.
  \item Das dvi-File ausdrucken.
\end{enumerate}
Zeitgem"a"se Betriebssysteme machen es m"oglich, da"s der Texteditor
und das Preview-Programm gleichzeitig in verschiedenen Fenstern 
"`ge"offnet"' sind; beim Durchlaufen des obigen Zyklus brauchen sie 
also nicht immer wieder von neuem gestartet werden.  Nur die 
wiederholte \LaTeX-Bearbeitung des Textes mu"s noch von Hand 
angesto"sen werden und l"auft ebenfalls in einem eigenen Fenster ab.
%Danach sollte das Preview-Programm -- im 
%Idealfall -- selbstt"atig das ver"anderte Ergebnis anzeigen; ansonsten 
%kennt es normalerweise einen Men"upunkt oder eine Schaltfl"ache, um das 
%ge"anderte Ausgabefile erneut zu laden und anzuzeigen.

Wie man auf die einzelnen Programme -- Editor, \LaTeX, Previewer,
Druckertreiber -- in einer bestimmten
Betriebs\-system\-umgebung zugreift, sollte im \local{}
beschrieben sein.



\section{Eingabefile}
Das Eingabefile f"ur \LaTeX{} ist ein Textfile. Es wird mit einem
Editor erstellt und ent"-h"alt sowohl den Text, der gedruckt
werden soll, als auch die Befehle, aus denen \LaTeX\ er"-f"ahrt,
wie der Text gesetzt werden soll.


\subsection{Leerstellen}
 
"`Unsichtbare"' Zeichen wie das Leerzeichen, Tabulatoren
und das Zeilenende werden von \LaTeX{}
einheitlich als Leerzeichen behandelt.  \emph{Mehrere}
Leerzeichen werden wie \emph{ein} Leerzeichen behandelt.   
Wenn man andere als die normalen Wort- und Zeilen"-ab"-st"ande
will, kann man dies also nicht durch die Eingabe von
zu"-s"atzlichen Leerzeichen oder Leerzeilen erreichen, sondern
nur mit entprechenden \LaTeX-Befehlen.

Eine Leerzeile zwischen Textzeilen bedeutet das Ende eines 
Absatzes.  \emph{Mehrere} Leerzeilen werden wie \emph{eine}
Leerzeile behandelt.
 
 
\subsection{\LaTeX-Befehle und Gruppen}
 
Die meisten \LaTeX-Befehle haben eines der beiden folgenden
Formate: Entweder sie beginnen mit einem Back\-slash~(\verb|\|)
und haben dann einen nur aus Buchstaben bestehenden Namen, der
durch ein oder mehrere Leerzeichen oder durch ein nachfolgendes
Sonderzeichen oder eine Ziffer beendet wird; oder sie bestehen
aus einem Back\-slash und genau einem Sonderzeichen oder einer
Ziffer.
Gro"s- und Kleinbuchstaben haben auch in Befehlsnamen
\emph{verschiedene} Bedeutung.
Wenn man nach einem Befehlsnamen eine Leerstelle erhalten will,
mu"s man~\verb|{}| zur Beendigung des Befehlsnamens oder einen
eigenen Befehl f"ur die Leerstelle verwenden.
\exa
  \renewcommand{\today}{35.~Mai 1998}  % to make sure that the
  % line breaks look good, regardless of the date of printing.
Heute ist der \today.
Oder: Heute ist der \today .
Falsch ist: Am \today regnet es.
Richtig: Am \today{} scheint die Sonne.
Oder: Am \today\ schneit es.
\exb
\begin{verbatim}
Heute ist der \today.
Oder: Heute ist der \today .
Falsch ist:
 Am \today regnet es.
Richtig:
 Am \today{} scheint die Sonne.
 Oder: Am \today\ schneit es.
\end{verbatim}
\exc
 
Manche Befehle haben Parameter, die zwischen geschwungenen
Klammern angegeben werden m"ussen.
Manche Befehle haben Parameter, die weggelassen oder zwischen
eckigen Klammern angegeben werden k"onnen.
Manche Befehle haben Varianten, die durch das Hin"-zu"-f"ugen
eines Sterns an den Befehlsnamen unterschieden werden.

Geschwungene Klammern k"onnen auch dazu verwendet werden, Gruppen
(groups) zu bilden.
Die Wirkung von Befehlen, die innerhalb von Gruppen oder
Umgebungen (environments) angegeben werden, endet immer mit dem
Ende der Gruppe bzw.\ der Umgebung.  Im obigen Beispiel
ist~\verb|{}| eine leere Gruppe, die au"ser der Beendigung des
Befehlsnamens \texttt{today} keine Wirkung hat.
 
\subsection{Kommentare}
 
Alles, was hinter einem Prozentzeichen (\verb|%|)
steht (bis zum Ende der Eingabezeile), wird von \LaTeX\ 
ignoriert.
Dies kann f"ur Notizen des Autors verwendet werden, die nicht
oder noch nicht ausgedruckt werden sollen.
\exa
Das ist ein % dummes
% Besser: ein lehrreiches <----
Beispiel.
\exb
\begin{verbatim}
Das ist ein % dummes
% Besser: ein lehrreiches <----
Beispiel.
\end{verbatim}
\exc
 
\subsection{Aufbau}
 
Der erste Befehl in einem \LaTeX-Eingabefile mu"s der Befehl
\begin{quote}
\verb|\documentclass|
\end{quote}
sein.
Er legt fest, welche Art von Schriftst"uck "uberhaupt erzeugt werden soll
(Bericht, Buch, Brief usw.).
Danach k"onnen weitere Befehle folgen, die f"ur das gesamte
Dokument gelten sollen.  Dieser Teil des Dokuments wird auch als 
\emph{Vorspann} oder \emph{Pr"aambel} bezeichnet.  Mit dem Befehl
\begin{quote}
\verb|\begin{document}|
\end{quote}
endet der Vorspann, und es 
beginnt das Setzen des Schrift"-st"ucks.
Nun folgen der Text und alle \LaTeX-Befehle, die das Ausdrucken
des Schrift"-st"ucks bewirken.
Die Eingabe mu"s mit dem Befehl
\begin{quote}
\verb|\end{document}|
\end{quote}
beendet werden.
Falls nach diesem Befehl noch Eingaben folgen, werden sie von
\LaTeX\ ignoriert.
 
Abbildung~\ref{mini} zeigt ein \emph{minimales} \LaTeX-File.
Ein etwas komplizierteres File ist in Abbildung~\ref{dokument}
skizziert.
 
\begin{figure}[hbp] %\small
\oben{6cm}
\begin{verbatim}
\documentclass{article}
\begin{document}
Small is beautiful.
\end{document}
\end{verbatim}
\unten
\caption{Ein minimales \LaTeX-File} \label{mini}
\end{figure}

% im folgenden Beispiel sollten Umlaute im Eingabefile auftreten!
\begin{figure}[hbtp] %\small
\oben{10cm}
\begin{flushleft}\ttfamily
% \verb+\NeedsTeXFormat{LaTeX2e}+\\
% hier unverst"andlich!
\verb+\documentclass[11pt,a4paper]{article}+\\
\verb+\usepackage[latin1]{inputenc}+\\
\verb+\usepackage{ngerman}+\\
\verb+\date{29. Februar 1998}+\\
\verb+\author{H.~Partl}+\\
\verb+\title{+"Uber kurz oder lang\verb+}+\\
\ \\
\verb+\begin{document}+\\
\verb+\maketitle+\\
\verb+\begin{abstract}+\\
Beispiel f"ur einen wissenschaftlichen Artikel\\
in deutscher Sprache.\\
\verb+\end{abstract}+\\
\verb+\tableofcontents+\\
\ \\
\verb+\section{Start}+\\
\ \\
Hier beginnt mein sch"ones Werk \dots\\
\ \\
\verb+\section{Ende}+\\
\ \\
\dots\ und hier endet es.\\
\ \\
\verb+\end{document}+\\[1\baselineskip]
\end{flushleft}
\unten
\caption{Aufbau eines Artikels} \label{dokument}
\end{figure}
 
 
\subsection{Dokumentklassen}\label{docsty}
 
Die am Beginn des Eingabefiles  mit
\begin{verse}
\verb|\documentclass[|\textit{optionen}\verb|]{|%
  \textit{klasse}\verb|}|
\end{verse}
definierte "`Klasse"' eines Dokumentes ent"-h"alt 
Vereinbarungen "uber 
das Layout und die logischen Strukturen, z.\,B.\ die 
Gliederungseinheiten (Kapitel etc.\@), 
die f"ur alle Dokumente dieses Typs gemeinsam sind.

Zwischen den geschwungenen Klammern \emph{mu"s} genau eine Dokumentklasse
angegeben werden.  Tabelle~\ref{docstyles} auf S.~\pageref{docstyles}
f"uhrt Klassen auf,
die in jeder \LaTeX-Installation existieren sollten.  
Im \local\ k"onnen weitere ver"-f"ug"-bare 
Klassen angegeben sein.  
 
Zwischen den eckigen Klammern \emph{k"onnen}, durch Kommata getrennt, 
eine oder mehrere Optionen f"ur das Klassenlayout
angegeben werden. Die wichtigsten Optionen sind in der 
Tabelle~\ref{options} auf S.~\pageref{options} an"-ge"-f"uhrt.
Das Eingabefile f"ur diese Beschreibung beginnt z.\,B.\ mit:
\begin{verse}
\verb|\documentclass[11pt,a4paper]{article}|
\end{verse}

\begin{table}[hbpt]
\caption{Dokumentklassen} \label{docstyles}
\oben{11cm}
\begin{ttdescription}%\small
\item [article] f"ur Artikel in wissenschaftlichen Zeitschriften,
  k"ur\-ze\-re Berichte u.\,v.\,a.
 
\item [report] f"ur l"angere Berichte, die aus mehreren Kapiteln
  bestehen, Diplomarbeiten, Dissertationen u.\,"a.
 
\item [book] f"ur B"ucher

\item[scrartcl, scrreprt, scrbook] sind Varianten der o.\,g. Klassen
mit besserer Anpassung an DIN-Papierformate und "`euro\-p"aische"'
Typographie. (Nicht "uberall vorhanden, siehe \local.)

\item [proc] f"ur Konferenzb"ande (Proceedings)

\item [letter] f"ur Briefe (siehe auch Abschnitt~\ref{briefe})

\item [slides] f"ur Folien. (Ersetzt das \textsc{Sli}\TeX-Format
  von \LaTeX~2.09.)
  
\end{ttdescription}
\unten
\end{table}

\begin{table}[hbpt]
\caption[Klassenoptionen]{Klassenoptionen (Alternativen sind durch \texttt{|}
  getrennt)} \label{options}
\oben{11cm}
\begin{ttdescription}%\small
\item [10pt|11pt|12pt] w"ahlt die normale Schriftgr"o"se des Dokuments aus.
  10\,pt hohe Schrift ist die Voreinstellung; diese Beschreibung benutzt 11\,pt.

\item[a4paper] f"ur Papier im DIN\,A4-Format. Ohne diese
  Option nimmt \LaTeX\ amerikanisches Papierformat an.
 
\item [fleqn] f"ur links"-b"undige statt zentrierte mathematische
  Gleichungen
 
\item [leqno] f"ur Gleichungsnummern links statt rechts von jeder
  numerierten Gleichung
 
\item [titlepage|notitlepage] legt fest, ob Titel und Zusammenfassung
  auf einer eigenen Seite erscheinen sollen.  \texttt{titlepage} ist
  die Voreinstellung f"ur die Klassen \texttt{report} und \texttt{book}.
 
\item [onecolumn|twocolumn] f"ur ein- oder zweispaltigen Satz.
 Die Voreinstellung ist immer \texttt{onecolumn}.  
 Die Klassen \texttt{letter} und \texttt{slides} kennen \emph{keinen}
 zweispaltigen Satz.
 
\item [oneside|twoside] legt fest, ob die Seiten f"ur ein- oder
  zweiseitigen  Druck gestaltet werden sollen.  
  \texttt{oneside} ist die Voreinstellung f"ur
  alle Klassen au"ser \texttt{book}.
  
\end{ttdescription}
\unten
\end{table}



\subsection{Pakete}\label{packages}
 
Mit dem Befehl
\begin{verse}
\verb:\usepackage[:\textit{optionen}\verb:]{:%
  \textit{pakete}\verb:}:
\end{verse}
k"onnen im Vorspann erg"anzende Makropakete (packages) geladen werden,
die das Layout der Dokumentklasse
modifizieren oder zus"atzliche Funktionalit"at bereitstellen.
Eine Auswahl von Paketen findet sich in der Tabelle~\ref{pack} 
auf S.~\pageref{pack}.
Das Eingabefile f"ur diese Beschreibung enth"alt beispielsweise:
\begin{verse}
\verb|\usepackage{german,latexsym,textcomp,alltt}|\\
\verb|\usepackage[dvips,draft]{graphics}|
\end{verse}


\begin{table}[htbp]
\caption{Pakete (eine Auswahl)}\label{pack}
\oben{11cm}
\begin{ttdescription}%\small
\setlength{\itemsep}{.5\itemsep plus1pt minus1pt}
%\item[a4] Anpassung an das Papierformat DIN\,A4, die "uber die
%  Klassenoption \texttt{a4paper} hinausgeht.
\item[alltt] Definiert eine Variante der \texttt{verbatim}-Umgebung
\item[amsmath, amssymb] Mathematischer Formelsatz mit erweiterten F"ahigkeiten,
  zus"atzliche mathematische Schriften und Symbole; Beschreibung siehe
  \cite{ch8}.
\item[array] Verbesserte und erweiterte Versionen der Umgebungen
  \texttt{array}, \texttt{tabular} und \texttt{tabular*}.
\item[babel] Anpassungen f"ur viele verschiedene Sprachen. Die
  ge"-w"ahlten Sprachen werden als Optionen angegeben.
\item[color] Unterst"utzung f"ur Farbdruck 
  (nur mit bestimmten Dru"ckertreibern); Beschreibung 
  siehe~\cite{grfguide} und~\cite{grfcomp}.
\item[dcolumn] Unterst"utzt auf Dezimaltrennzeichen ausgerichtete
  Spalten in den Umgebungen \texttt{array} und \texttt{tabular}
\item[fontenc] Erlaubt die Verwendung von Schriften mit
  unterschiedlicher Kodierung (Zeichenvorrat, Anordnung).
\item[fancyhdr] Flexible Gestaltung von Kopf- und Fu"szeilen.
\item[german, ngerman] Anpassungen f"ur die deutsche Sprache in
  traditioneller und neuer Rechtschreibung.
\item[graphics] Einbindung von extern erzeugten Graphiken.
  Die umfangreichen M"og"-lichkeiten dieses Pakets werden 
  in~\cite{grfguide} und~\cite{grfcomp} beschrieben.
\item[inputenc] Deklaration der Zeichen"-kodierung im
  Eingabe"-file.
\item[latexsym] Erlaubt einige besondere Symbole wie~\(\Box\),
  die mit \LaTeX~2.09 standardm"a"sig verf"ugbar waren.
\item[longtable]
  f"ur Tabellen "uber mehrere Seiten mit automatischem Seiten"-umbruch.
\item[makeidx] Unterst"utzt das Erstellen eines Index.
\item[multicol] Mehrspaltiger Satz mit Kolumnenausgleich.
\item[showkeys] Druckt die Namen aller verwendeten \verb:\label:s,
  \verb:\ref:s und \verb:\pageref:s im Text aus.
\item[textcomp] Bindet Schriften mit zus"atzlichen Textsymbolen ein.
%\item[verbatim] Flexible Erweiterung der \texttt{verbatim}-Umgebung.
\end{ttdescription}
\unten
\end{table}


\subsection{Eingabezeichensatz}\label{inputenc}

Bei jedem \LaTeX-System d"urfen mindestens die folgenden
Zeichen zur Eingabe von Text verwendet werden:
\begin{quote}
  \ttfamily
  a\dots z A\dots Z 0\dots 9 \\
  . : ; , ? ! ` ' ( ) [ ] - / * @ + =
\end{quote}
Die folgenden Eingabezeichen haben f"ur \LaTeX{} eine Spezialbedeutung
oder sind nur innerhalb von mathematischen Formeln erlaubt:
\begin{quote}
\verb.$ & % # _ { }  ~  ^  "  \  | < >.
\end{quote}
F"ur Zeichen, die "uber obige Liste hinausgehen, beispielsweise die Umlaute,
sind je nach Betriebssystem des verwendeten Computers 
unterschiedliche Kodierungen in Gebrauch.  Damit auch diese Zeichen im 
Eingabe\-file benutzt werden d"urfen,  mu"s man das Paket 
\texttt{inputenc} laden und dabei die jeweilige Kodierung als 
Option angeben: \verb:\usepackage[:\textit{codepage}\verb:]{inputenc}:.
M"ogliche Angaben f"ur \textit{codepage} sind u.\,a.:
\begin{ttdescription}
  \item[latin1] ISO Latin-1, gebr"auchlich auf \textsc{Unix}-Systemen und VMS
  \item[cp850] IBM Codepage 850, "ublich unter OS/2
  \item[ansinew] Latin-1 mit Erweiterungen \`a la \textsc{Windows}
  \item[cp437de] IBM Codepage 437, "ublich unter DOS
  \item[applemac] \textsc{Macintosh}-Kodierung
\end{ttdescription}
(Es ist m"oglich, da"s bestimmte \LaTeX-Implementierungen
diese F"ahigkeit nicht unterst"utzen; ziehen Sie dazu den \local{} zu Rate.)
Falls \LaTeX{} ein eingegebenes Zeichen nicht darstellen
kann, was meist f"ur die sogenannten "`Pseudografik-Zeichen"' 
gilt,  bekommt man eine entsprechende Fehlermeldung.
%\footnote{Bis zur 
%\LaTeX-Version \texttt{<1998/06/01>} war diese Fehlermeldung 
%leider relativ unverst"andlich (\texttt{unknown control sequence}).}.
Auch sind manche Zeichen nur im Text, andere nur in mathematischen 
Formeln erlaubt.

Man beachte, da"s der in der \emph{Ausgabe} darstellbare Zeichenvorrat 
von \LaTeX{} nicht davon abh"angt, welche Zeichen als \emph{Eingabe} erlaubt 
sind:
F"ur jedes "uberhaupt darstellbare Zeichen -- also auch diejenigen, die
nicht im Zeichensatz des jeweiligen Betriebssystems enthalten sind --
gibt es einen 
\LaTeX-Befehl oder eine Ersatzdarstellung, die ausschlie"slich mit 
ASCII-Zeichen auskommt.  N"aheres dar"uber erfahren Sie
in Abschnitt \ref{spezial}.



\endinput
